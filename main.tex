
\documentclass{LU}

\title{Bakalaura darba nosaukums}	   
\thesistype{Bakalaura darbs}
\author{Vārds Uzvārds}
\studentid{xy00000}
\supervisor{Vārds Uzvārds}
\university{Latvijas Universitāte}
\faculty{Fakultātes nosaukums}
\location{Rīga}


% apzīmējumu saraksts
\usepackage[toc,acronym]{glossaries} 
\makeglossaries
\newglossaryentry{BS}{name={BS}, description={\textit{bull-shit}. A blatant lie, a fragrant untruth, an obvious falicy}}

\begin{document}

\maketitle

\begin{abstract}
This document is example on how to use latex and \textit{lu.cls} classs.
\end{abstract}
 

\selectlanguage{english}
\begin{abstract}
This document is example on how to use latex and \textit{lu.cls} classs.
\end{abstract}
\selectlanguage{latvian}

\pagenumbering{gobble} 

\tableofcontents

%------------------------------------------------APZĪMĒJUMI---------------------------------------------------------

\printglossary[type=main,title={Apzīmējumu saraksts},toctitle={Apzīmējumu saraksts}]

\pagenumbering{arabic} % sākam numurēt lapas no apzīmējumu saraksta (3. pielikums iekš LU 03.02.2012, 1/38 ) 

\chapter*{Ievads} % * nepieliks numuru pie nosaukuma
\addcontentsline{toc}{chapter}{Ievads}
\pagestyle{plain}
Šis ir dokumenta ievads.
Visu dokumentu ir vieglāk pārvaldīt ja sadala daļās un katru noglabā savā failā.



%------------------------------------------------DARBS--------------------------------------------------------------

\chapter{Problēmas pamatnostādne un literatūras apskats}
\section{Problēmas pamatnostādne}
Šajā nodaļa mēs apskatam pētāmās problēmas teorētisko pusi.
Veicam daudz $copy + paste + alter \vee trnslate$.
Galvenais ir salikt labi daudz atsauču \cite{idk}.
Patiesībā katru teikumu varam uztver kā BS un atsauces ir veids kā pateikt, ka tas nav \gls{BS}.

\section{Saistība ar citiem pētījumiem}

Te vajag aprakstīt saistību?
Vai šāda nodaļa vispār ir vajadzīga?

\chapter{Risinājumus}
\section{Motivācija}	
Motivācija risinājumam?

\section{Analīze}
Te būs skaita analīze

%----------------------------------------------SECINĀJUMI----------------------------------------------------------

\chapter*{Secinājumi}
\addcontentsline{toc}{chapter}{Secinājumi}
Secinājumi - kas ieguldīts, vai mērķis ir sasniegts (jo ir izpildīti uzdevumi)

%---------------------------------------------LIETERATŪRA----------------------------------------------------------
\renewcommand{\bibname}{Izmantotā literatūra un avoti}
\bibliographystyle{alpha} % nekur nav minēts kādam jābūt atsaucu noformējumam
\bibliography{main}
\addcontentsline{toc}{chapter}{Izmantotā literatūra un avoti}


%----------------------------------------------PIELIKUMS----------------------------------------------------------

\begin{appendices}
\chapter*{Pielikums}
\renewcommand{\thesection}{\arabic{section}}
\titleformat{\section}{\normalfont\large\bfseries}{\thesection. pielikums.}{1em}{}

\section{skaists pielikums, lai vairāk lapu}

;)
\end{appendices}

\end{document}
