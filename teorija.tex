\section{Problēmas pamatnostādne}
Šajā nodaļa mēs apskatam pētāmās problēmas teorētisko pusi.
Veicam daudz $copy + paste + alter \vee trnslate$.
Galvenais ir salikt labi daudz atsauču \cite{idk}.
Patiesībā katru teikumu varam uztver kā BS un atsauces ir veids kā pateikt, ka tas nav \gls{BS}.

\section{Saistība ar citiem pētījumiem}

Te vajag aprakstīt saistību?
Vai šāda nodaļa vispār ir vajadzīga?